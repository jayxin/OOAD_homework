\documentclass[../main.tex]{subfiles}
\graphicspath{{\subfix{../figures/}}}
%
\begin{document}
%
\section{需求说明}
\subsection{概述}
以``智慧食堂''为题,利用面向对象分析与设计方法,建立智慧食堂点餐服务系统的
UML 模型(含用例模型、静态模型和动态模型等),并实现该系统。
%
\subsection{具体需求}
某大学食堂希望改变当前的堂食服务模式,将以自主点餐、自主取餐、自动结算的方式
进一步提升食堂运转效率,解决食堂人员成本高、用餐高峰期拥堵等问题。

新的食堂运转模式如下:

用餐人员进入食堂后,不用去取餐盘,也不用去窗口点餐,而是直接寻找到空闲的餐桌
坐下。每个餐桌上有若干个餐位,每个餐位的桌面上都有一个二维码。用餐人员用食堂 APP
扫该二维码进行点餐。如果用餐人员未注册过账号,需要先注册。食堂 APP 上把菜品分类
显示,包括猪肉类、牛肉类、羊肉类、水产类、豆制品类、蔬菜类、鸡蛋类、主食类。每个
分类下显示具体的菜品名称、配料、价格。用餐人员可在 APP 中点菜,选中菜品时默认份
数为 1 份,也可调整数量。选择菜品时,APP 上实时显示已选菜品总份数和价格。点餐的过
程中可以终止本次点餐。点餐完可进行结算,如果账户余额充足,则点餐成功;如果余额不
足,则需要先在线充值。点餐成功后,用餐人员可坐在餐位上等待配餐完成,在食堂 APP
上可以看到他的等待排序(按点餐提交的时间,先来先服务)。

食堂工作人员的主要职责是根据点餐结果进行配餐。工作人员的工作台上有一台带触摸
屏和打印机的一体机,屏上显示分配给他的所有点餐记录。每条点餐记录包括餐位号、菜品
及其份数。工作人员按顺序配餐,他先点击``开始''按钮(按钮上的文字切换为``完成''),
为点餐记录列表的第一条点餐记录配餐。此时用餐人员在食堂 APP 上可以看到工作人员已
经开始为他配餐。工作人员配餐完币,点击``完成''按钮(按钮上的文字切换为``开始''),
此时用餐人员在食堂 APP 上可以看到配餐完成,一体机也将打印出一张含有餐位号及点餐
人员 ID 的单据。用餐人员自行到取餐区取餐。

食堂工作人员可以通过该系统进行查询分析,包括:各菜品的受欢迎程度、食堂各时间
段的用餐人数、平均点餐配餐时间、工作人员的工作量、累计收入等。
%
\end{document}
